% THIS IS SIGPROC-SP.TEX - VERSION 3.1
% WORKS WITH V3.2SP OF ACM_PROC_ARTICLE-SP.CLS
% APRIL 2009
%
% It is an example file showing how to use the 'acm_proc_article-sp.cls' V3.2SP
% LaTeX2e document class file for Conference Proceedings submissions.
% ----------------------------------------------------------------------------------------------------------------
% This .tex file (and associated .cls V3.2SP) *DOES NOT* produce:
%       1) The Permission Statement
%       2) The Conference (location) Info information
%       3) The Copyright Line with ACM data
%       4) Page numbering
% ---------------------------------------------------------------------------------------------------------------
% It is an example which *does* use the .bib file (from which the .bbl file
% is produced).
% REMEMBER HOWEVER: After having produced the .bbl file,
% and prior to final submission,
% you need to 'insert'  your .bbl file into your source .tex file so as to provide
% ONE 'self-contained' source file.
%
% Questions regarding SIGS should be sent to
% Adrienne Griscti ---> griscti@acm.org
%
% Questions/suggestions regarding the guidelines, .tex and .cls files, etc. to
% Gerald Murray ---> murray@hq.acm.org
%
% For tracking purposes - this is V3.1SP - APRIL 2009

\documentclass{edm_template}

\usepackage{tikz}
\usetikzlibrary{bayesnet}
\usetikzlibrary{arrows}
\usepackage{enumitem}
\usepackage{amsmath}

% \usepackage[dvipsnames]{xcolor}

\providecommand{\am}[1]{{\color{blue} [AM: #1]}}
\providecommand{\ng}[1]{{\color{red} [NG: #1]}}

\begin{document}

\title{Latent Variable Models of Enrollment for Course Planning and Understanding}
%\subtitle{[Extended Abstract]
%\titlenote{A full version of this paper is available as
%\textit{Author's Guide to Preparing ACM SIG Proceedings Using
%\LaTeX$2_\epsilon$\ and BibTeX} at
%\texttt{www.acm.org/eaddress.htm}}}
%
% You need the command \numberofauthors to handle the 'placement
% and alignment' of the authors beneath the title.
%
% For aesthetic reasons, we recommend 'three authors at a time'
% i.e. three 'name/affiliation blocks' be placed beneath the title.
%
% NOTE: You are NOT restricted in how many 'rows' of
% "name/affiliations" may appear. We just ask that you restrict
% the number of 'columns' to three.
%
% Because of the available 'opening page real-estate'
% we ask you to refrain from putting more than six authors
% (two rows with three columns) beneath the article title.
% More than six makes the first-page appear very cluttered indeed.
%
% Use the \alignauthor commands to handle the names
% and affiliations for an 'aesthetic maximum' of six authors.
% Add names, affiliations, addresses for
% the seventh etc. author(s) as the argument for the
% \additionalauthors command.
% These 'additional authors' will be output/set for you
% without further effort on your part as the last section in
% the body of your article BEFORE References or any Appendices.

\numberofauthors{5} %  in this sample file, there are a *total*
% of EIGHT authors. SIX appear on the 'first-page' (for formatting
% reasons) and the remaining two appear in the \additionalauthors section.
%
\author{
% You can go ahead and credit any number of authors here,
% e.g. one 'row of three' or two rows (consisting of one row of three
% and a second row of one, two or three).
%
% The command \alignauthor (no curly braces needed) should
% precede each author name, affiliation/snail-mail address and
% e-mail address. Additionally, tag each line of
% affiliation/address with \affaddr, and tag the
% e-mail address with \email.
%
\alignauthor Nate Gruver \\
       \email{ngruver@cs.stanford.edu}
\alignauthor Ali Malik \\
       \email{malikali@cs.stanford.edu}
\alignauthor Brahm Capoor
       \email{brahm@cs.stanford.edu}
\and
\alignauthor Chris Piech\\
       \email{piech@cs.stanford.edu}
\alignauthor Andreas Paepcke \\
       \email{paepcke@cs.stanford.edu}
}
% There's nothing stopping you putting the seventh, eighth, etc.
% author on the opening page (as the 'third row') but we ask,
% for aesthetic reasons that you place these 'additional authors'
% in the \additional authors block, viz.

% Just remember to make sure that the TOTAL number of authors
% is the number that will appear on the first page PLUS the
% number that will appear in the \additionalauthors section.

\maketitle
\begin{abstract}

\end{abstract}

%% A category with the (minimum) three required fields
%\category{H.4}{Information Systems Applications}{Miscellaneous}
%%A category including the fourth, optional field follows...
%\category{D.2.8}{Software Engineering}{Metrics}[complexity measures, performance measures]
%
%\terms{Theory}

\keywords{Enrollment modeling, mixture models, hidden Markov model} % NOT required for Proceedings

\section{Introduction}

Understanding the enrollment decisions of students in post-secondary institutions is central to the success of university administrators, researchers, and advisors. Despite this, students and staff rely primarily on heuristically constructed lists and complex mental models shared collectively through word-of-mouth for the bulk of course planning. Although work has been devoted to predicting course enrollments, few studies have come close to replicating the robust framework experienced students hold within their head---one that is capable of imagining counterfactual paths, inferring intentions, or detecting anomalies in various course planning scenarios.


\am{To handle these more nuanced tasks, we propose a generative model of course enrollment. Link with motivating use cases below}

In this work we focus on three motivating use cases for such a model:
\vspace{-3mm}
\begin{enumerate}[noitemsep,topsep=0pt]
	\item A student prototyping their set of future enrollments based on common paths, i.e. queries of the form
	$$\text{P}(\text{Course}_i | \text{Course}_{j \neq i})$$ 
	\item An advisor wondering if their advisee is on track, i.e. queries of the form
	$$P(\text{Student's enrollments} | \text{All other student enrollments})$$
	\item An administrator wondering about optimal resource allocation, i.e. queries of the form
	$$P(\text{All future enrollments} | \text{All past enrollments})$$
\end{enumerate}

\subsection{Prior Work}

Much of the prior work on enrollment modeling in the university setting is dedicated purely to predictive models, both of future course enrollment \cite{kardan2013prediction}\cite{nandeshwar2009enrollment}\cite{song1993new} and academic performance \cite{kovacic2010early} \cite{hlosta2017ouroboros}. The rise of MOOCs has also been accompanied by a flurry of work predicting and tracking online student success and engagement through course decisions \cite{balakrishnan2013predicting}\cite{Gardner2018StudentSP}\cite{AlShabandar2018TheAO}. The techniques used in these predictive tasks range from simple probabilistic models to deep neural networks. A related body of work is that on recommender systems for course enrollment, with similar neural network~\cite{Jiang2018GoalbasedCR} and probabilistic approaches~\cite{Khorasani2016AMC}. 

Another significant branch of enrollment modeling research is unsupervised learning for clustering, classification, and feature learning. Some authors focus on clustering courses with topic modeling and embedding techniques~\cite{Motz2018FindingTI}~\cite{Pardos2018AMO} while others focus instead on students--using patterns of enrollments to extract meaningful types via vector quanitization and graph analysis~\cite{Zeidenberg2011TheCO}~\cite{Slim2016TheIO}. Some even treat coenrollments as a network and extract graph-theoretic properties for representation learning~\cite{gardner2018coenrollment}~\cite{Wang2017AnalyzingCC}. 

The primary limitations found in the prior work is a lack of flexibility. Strong predictive models can be trained, but they are most discriminative instead of generative and thus lack the ability to answer nuanced inference queries about enrollment decisions conditioned on both past and future enrollments. While some approaches do achieve this level of expressiveness~\cite{Jiang2018GoalbasedCR}, they lack the ability to simulate out different high-probability sequences of enrollments--a functionality that allows students to more thoroughly prototype. 

Similarly the unsupervised approaches while often effective often don't allow for estimation of key probabilistic metrics like the posterior class probability or log likelihood of a sample. Adapting the more general probabilistic approach presented here opens the door for a much more diverse set of analyses within a shared statistical framework.

\section{Background}

There is a rich history of probabilistic modeling of categorical data in the natural language processing literature. Here we use these techniques to model enrollments as analogues of natural language tokens, with each student comprising either a set or sequence of sets of enrollments. Of particular relevance are mixture and hidden Markov models which can be used to capture the typography of students and the sequential nature of their enrollments. 

\subsection{Mixture Models}

Among the simplest and most common mixture models used in generative modeling of categorical variables with strong correlation is Latent Dirichlet allocation (LDA). In this mixture model, the hidden variable, or topic, is drawn from a multinomial distribution and then words are sampled conditioned on the topic:
\begin{align*}
\theta &\sim \text{Dir}(\alpha) \\
N &\sim \text{Poisson}(\xi) \\
i &\in [1,N] \\
\text{ topic } z_i &\sim \text{Multinomial}(\theta) \\
\text{ word } w_i &\sim \text{Multinomial}(w_i | z_i, \beta) \\ 	
\end{align*}

\vspace{-7mm}
The Gaussian mixture model similarly relies on a latent categorical variable, but with a multivariate normal distribution as the conditional distribution from which observed samples are drawn. In particular
\vspace{-7mm}

\begin{align*}
h_i &\sim \text{Multinomial}(\theta) \\
X_j &\sim \mathcal{N}(X_j | h_i ; \mu_i, \Sigma_i) \\
\end{align*}

\vspace{-7mm}
A graphical representation of a generalize mixture model with hidden categorical variable $h$ and observed variable $X$ is shown in Figure \ref{fig:mixture_model}. 

\subsection{Contextual Mixture Model}

\section{Models}

\begin{figure}
	\centering
	\tikz{
	% nodes
	\node[obs] (X) {$X_i$};%
	\node[latent,above=of X] (h) {$h_i$}; %
	% plate
	\plate [inner sep=.25cm,yshift=.2cm] {plate1} {(X)(h)} {$N$}; %
	% edges
	\edge {h} {X}  }
	\caption{\label{fig:mixture_model} Graphical representation of mixture model using plate notation}
\end{figure}


\section{Results}



\section{Conclusions}


%ACKNOWLEDGMENTS are optional
\section{Acknowledgments}

%
% The following two commands are all you need in the
% initial runs of your .tex file to
% produce the bibliography for the citations in your paper.
\bibliographystyle{abbrv}
\bibliography{sigproc}  % sigproc.bib is the name of the Bibliography in this case
% You must have a proper ".bib" file
%  and remember to run:
% latex bibtex latex lvuatex
% to resolve all references
%
% ACM needs 'a single self-contained file'!
%
%APPENDICES are optional
%\balancecolumns
% \appendix
% %Appendix A
% \section{Headings in Appendices}
% The rules about hierarchical headings discussed above for
% the body of the article are different in the appendices.
% In the \textbf{appendix} environment, the command
% \textbf{section} is used to
% indicate the start of each Appendix, with alphabetic order
% designation (i.e. the first is A, the second B, etc.) and
% a title (if you include one).  So, if you need
% hierarchical structure
% \textit{within} an Appendix, start with \textbf{subsection} as the
% highest level. Here is an outline of the body of this
% document in Appendix-appropriate form:
% \subsection{Introduction}
% \subsection{The Body of the Paper}
% \subsubsection{Type Changes and  Special Characters}
% \subsubsection{Math Equations}
% \paragraph{Inline (In-text) Equations}
% \paragraph{Display Equations}
% \subsubsection{Citations}
% \subsubsection{Tables}
% \subsubsection{Figures}
% \subsubsection{Theorem-like Constructs}
% \subsubsection*{A Caveat for the \TeX\ Expert}
% \subsection{Conclusions}
% \subsection{Acknowledgments}
% \subsection{Additional Authors}
% This section is inserted by \LaTeX; you do not insert it.
% You just add the names and information in the
% \texttt{{\char'134}additionalauthors} command at the start
% of the document.
% \subsection{References}
% Generated by bibtex from your ~.bib file.  Run latex,
% then bibtex, then latex twice (to resolve references)
% to create the ~.bbl file.  Insert that ~.bbl file into
% the .tex source file and comment out
% the command \texttt{{\char'134}thebibliography}.
% % This next section command marks the start of
% % Appendix B, and does not continue the present hierarchy
% \section{More Help for the Hardy}
% The acm\_proc\_article-sp document class file itself is chock-full of succinct
% and helpful comments.  If you consider yourself a moderately
% experienced to expert user of \LaTeX, you may find reading
% it useful but please remember not to change it.
% \balancecolumns
% % That's all folks!
\end{document}
